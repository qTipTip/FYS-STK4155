\documentclass[article, a4paper, oneside, 12pt]{memoir}
\linespread{1.25}
%\setlength{\parskip}{1em}
%\setlength{\parindent}{0em}

\usepackage[]{mathtools}
\usepackage[]{amssymb}
\usepackage[]{libertine}
\usepackage[]{libertinust1math}
\usepackage[T1]{fontenc}
\usepackage[]{microtype}
\usepackage[]{geometry}
\usepackage[]{amsthm}
\usepackage[]{thmtools}
\usepackage[usedvipsnames]{xcolor}
\usepackage[]{bm}
\usepackage[]{amsmath}
\usepackage[]{commath}
\usepackage[noabbrev, capitalize]{cleveref}


\declaretheorem[style=remark, name=Remark,
mdframed={
}]{remark}

\title{\textsc{Regression And Resampling Techniques \\
  FYS-STK4155 \\
Assignment One}}
\author{Ivar Haugal{\o}kken Stangeby}

\newcommand{\x}{\bm{x}}
\newcommand{\C}{\mathbb{C}}
\newcommand{\X}{\bm{X}}
\newcommand{\diag}[1]{\mathrm{diag}(#1)}
\newcommand{\MSE}[1]{\mathrm{MSE}(#1)}
\newcommand{\R}[1]{\mathrm{R}^2(#1)}

\newcommand{\mat}[1]{\bm{#1}}
\newcommand{\y}{\bm{y}}
\newcommand{\data}{\bm{\Omega}}
\newcommand{\cost}{\mathcal{C}}
\newcommand{\N}{\mathcal{N}}
\newcommand{\basis}{\mathcal{B}}
\newcommand{\set}[1]{\left\{\, #1 \right\}}

\begin{document}
  \maketitle 

  \chapter{Introduction}
  
  Regression analysis is the act of estimating relationships among variables.
  In this project, we study various regression methods in more detail. In
  particular, we compare the \emph{ordinary least squares} (OLS) method with
  the \emph{Ridge regression} and \emph{Lasso regression} techniques. As we
  shall see, these three methods are all variations over the same theme. We
  start by testing the methods on noisy data sampled from a function known as
  \emph{Franke's} bivariate test function, which has been in widespread use in
  the evaluation of interpolation and data fitting techniques. Finally, we run
  regression on real terrain data, comparing the aforementioned methods.

  \chapter{Regression Techniques}

  In general, the goal of regression analysis is to fit a \emph{model function}
  \( f(\x, \beta) \) to a set of \( n \) data points \( \data = (\x_i,
  y_i)_{i=1}^n \). A simple example is that of a linear polynomial with two
  parameters:
  \begin{equation}
    f(x, \beta) = \beta_0 + \beta_1 x.
  \end{equation}
  The \emph{model parameters} \( \beta \) are determined in order to minimize a
  suitable \emph{cost function} \(\cost(\data, \beta)\) which measures to which
  extent the model function manages to capture trends in the data \( \data \).
  It is the choice of cost function \( \matcal{C}(\data, \beta)\) which
  distinguishes  the three regression techniques, OLS, Lasso regression, and
  Ridge regression.

  \begin{remark}
    It is often assumed a priori that the data is infact generated from a noisy
    model, such that each \( y_i \) can be described as
    \begin{equation}
      y_i = f(\x_i) + \varepsilon_i
    \end{equation}
    where each \( \varepsilon_i \sim \N(0, \sigma^2) \) is normally distributed
    with zero mean and variance \( \sigma^2 \). This assumption on the error
    gives rise to what is known as \emph{general linear models}.
  \end{remark}


  \paragraph{The design matrix.} We are often interested in finding the best
  model function in a specific function space.  Assuming this space has a basis
  \( \basis = \set{\varphi_i}_{i=1}^M\), we may write our model function in
  terms of the basis functions and the model parameters as
  \begin{equation}
    \label{eq:gen_lin_mod}
    f(\x, \beta) \coloneqq \sum_{i = 1}^M \beta_i\varphi_i(\x).
  \end{equation}
  As an example, if we were to use the space \( \Pi_2^2 \) of bi-quadratic
  polynomials, our basis functions would be
  \begin{equation}
    \basis = \set{1, x, x^2, y, y^2, xy}.
  \end{equation}
  With this formulation, we can represent the approximation \( \hat{\y} \) to
  \( \y \) as a matrix product 
  \begin{equation}
    \label{eq:matrix_prod}
    \hat{\y} = \X\beta
  \end{equation}
  where \( \X \) is the \emph{design matrix} defined by components \(\X_{ij} =
  \varphi_j(\x_i)\), and \( \beta = [\beta_1, \ldots, \beta_M] \) is the model
  parameters. 

  \paragraph{Performance metrics.}
  In order to evaluate the how accurately the model function \( f(\data, \beta)
  \) captures trends in the target data \( \y \), a few standard performance
  metrics are  used.  Firstly, the \emph{mean squared error}:
  \begin{equation}
    \MSE{\y, \hat{\y}} \coloneqq \frac{1}{n} \sum_{i=1}^n (y_i - \hat{y}_i)^2
  \end{equation}
  which simply averages the squared error over all samples and estimates.
  Secondly, we have the \emph{coefficient of determination} or
  \emph{\(R^2\)-score}:
  \begin{equation}
    \R{\y, \hat{y}} \coloneqq 1 - \frac{\sum_{i=1}^n (y_i -
    \hat{y}_i)^2}{\sum_{i=1}^n (y_i - \overline{y})^2}.
  \end{equation}
  The \(R^2\)-score measures much of the variation in \( \y \) which can be
  attributed to a simple linear relation between \( \x \) and \( \y \). It is a
  ratio, thus a value of one tells us that all the variation in the data can be
  attributed to an approximate linear relationship between the data \(\x\) and
  the response variable \( \y \). A value of zero indicates that a non-linear
  model may be preferrable.
  

  \section{Ordinary Least Squares}
  
  One common cost function is the one which involves the sum of squared
  residuals (or squared errors):
  \begin{equation}
    \label{eq:mse}
    \cost(\data, \beta) = \frac{1}{n}\sum_{i = 1}^n \varepsilon_i^2 \coloneqq \frac{1}{n}\sum_{i=1}^n (y_i - \hat{y}_i)^2
  \end{equation}
  where \( \hat{y}_i \coloneqq f(\x_i, \beta) \). The method involving the
  minimization of this specific cost function is known as \emph{least squares}.
  With the matrix notation from above in mind, we can also write the cost
  function as
  \begin{equation}
    \cost(\data, \beta) = \frac{1}{n} (\y - \hat{\y})^T (\y - \hat{\y}).
  \end{equation}

  

  \paragraph{Optimizing the parameters in OLS.}
  
  Assume now that our model is of the form given in \cref{eq:gen_lin_mod} and
  that our cost function is the mean squared error defined in \cref{eq:mse}. We
  are interested in finding the parameters \( \beta \) that minimize the cost
  function \( \cost(\data, \beta) \). Since \( \cost(\data, \beta) \) is
  convex, it suffices to differentiate with respect to \( \beta \) and equating
  to zero. 
  
  We have that
  \begin{align}
    \dpd{\cost(\data, \beta)}{\beta} = \X^T(\y - \X\beta), 
  \end{align}
  and equating this to zero yields the familiar \emph{normal equations}
  \begin{equation}
    \X^T \y = \X^T\X\beta.
  \end{equation}
  If the matrix \( \X^T\X \) is invertible, we may obtain the solution by
  direct numerical matrix inversion. In this case, the optimal model parameters
  are found directly by
  \begin{equation}
    \label{eq:optimal}
    \beta = (\X^T\X)^{-1}\X^T\y.
  \end{equation}
  However, these matrices may be ill-conditioned when the number of equations
  are very large, and it is therefore common to apply approximate solvers for
  the inverse, for instance using low-rank approximation based on
  \textsc{SVD}-decomposition, which we will briefly turn to in the following.

  \paragraph{Singular value decomposition.}
  
  Recall that any matrix \( \mat{A} \in \C^{n, m} \) can be decomposed as
  \begin{equation}
    \mat{A} = \mat{U} \mat{\Sigma}\mat{V}^T,
  \end{equation}
  where \( \mat{U} \) and \( \mat{V} \) are comprised of the eigenvectors of \(
  \mat{A}\mat{A}^T \) and \( \mat{A}^T\mat{A} \) respectively. As these
  eigenvectors are orthonormal, it follows that both \( \mat{U} \) and \(
  \mat{V} \) are unitary matrices. Furthermore, \( \mat{\Sigma} \) is a matrix 
  \begin{equation}
    \mat{\Sigma} = \begin{bmatrix}
      \Sigma_1 & 0 \\
       0 & 0
    \end{bmatrix},
  \end{equation}
  where \( \Sigma_1 \) is a square diagonal matrix of size \( r \times r \)
  with the non-zero singular values of \( \mat{A} \). The integer \( r \) is
  the \emph{rank} of \( \mat{A} \). As the matrix \( \Sigma \) is mostly
  containing zeros, the information stored in \( \mat{A} \) is attributed to
  only some parts of \( \mat{U} \) and \( \mat{V} \). We can remove the
  redundant parts, and more compactly express \( \mat{A} \) as \( \mat{A} =
  \mat{U}_1 \mat{\Sigma}_1 \mat{V}_1^T \) without loss of accuracy in the
  decomposition. Here, \( \mat{U}_1 \) is \( m \times r \) and \( \mat{V}_1 \)
  is \( n \times r \).

  Applying the singular value decomposition to the matrix \(\X = \mat{U}
  \mat{\Sigma} \mat{V}^T\) we can analyze the expression for the prediction \( \hat{\y} \)
  in terms of the matrix \( \X^T \X \).  First of all, we have that
  \begin{align}
    \begin{split}
    \X^T \X &= \mat{V}\mat{\Sigma}^T\mat{U}^T\mat{U}\mat{\Sigma} \mat{V}^T\\
            &= \mat{V}\mat{D}\mat{V}^T, 
    \end{split}
  \end{align}
  where we have used that \( \mat{U}\mat{U}^T = \mat{I} \) and defined \(
  \mat{D} = \diag{\sigma_1^2, \ldots, \sigma_n^2} \).  Thus, plugging this into
  \cref{eq:optimal,eq:matrix_prod}, we obtain the expression
  \begin{align}
    \begin{split}
      \hat{\y} = \X \beta = \X (\mat{V}\mat{D}\mat{V}^T)^{-1} \mat{X}^T \y.
    \end{split}
  \end{align}
  Finally, substituting \( \X = \mat{U}\mat{\Sigma}\mat{V}^T \) and noting that
  diagonal matrices always commute, we end up with
  \begin{align}
    \begin{split}
      \hat{\y} = \mat{U}\mat{U}^T \mat{y}.
    \end{split}
  \end{align}


\end{document} 
